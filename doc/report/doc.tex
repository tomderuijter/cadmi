% Needed packages
\documentclass[a4paper, 10pt, english, onecolumn]{article}
\usepackage[english]{babel}
\usepackage[cm]{fullpage}
\usepackage{cite}
\usepackage{anysize}
\usepackage[compact]{titlesec}
\usepackage{hyperref}

\usepackage{amssymb,amsmath}

% Margins & Headers
\marginsize{2.5cm}{2.5cm}{3.0cm}{2.0cm}
\columnsep 0.4in
\footskip 0.4in 
\usepackage{changepage}

% E-mail formatting
\usepackage{color,hyperref}
    \catcode`\_=11\relax
    \newcommand\email[1]{\_email #1\q_nil}
    \def\_email#1@#2\q_nil{
      \href{mailto:#1@#2}{{\emailfont #1\emailampersat #2}}
    }
    \newcommand\emailfont{\sffamily}
    \newcommand\emailampersat{{\color{red}\small@}}
    \catcode`\_=8\relax 

% List modifications
\newenvironment{packed_item}{
\begin{itemize}
  \setlength{\itemsep}{1pt}
  \setlength{\parskip}{0pt}
  \setlength{\parsep}{0pt}
}{\end{itemize}}

\newenvironment{packed_enum}{
\begin{enumerate}
  \setlength{\itemsep}{1pt}
  \setlength{\parskip}{0pt}
  \setlength{\parsep}{0pt}
}{\end{enumerate}}

% ############## End Macros ##############

% Title
\title{\fontfamily{phv}\selectfont{Another ad-hoc CAD system}}
\author{
  \textbf{W. Kanters} - \href{mailto:kantersw@gmail.com}{kantersw@gmail.com} \\
  \textbf{T. de Ruijter} - \href{mailto:t.deruijter@student.ru.nl}{t.deruijter@student.ru.nl}\\
}

\date{\fontfamily{ptm}\selectfont{\small{\bfseries{\today - Radboud
Universiteit Nijmegen}}}\\[0.5cm]\rule{\linewidth}{0.3mm}}

\begin{document}

\maketitle

\setlength{\parindent}{0.0cm}
\setlength{\parskip}{0.25cm}

\section{Introduction}
% 1 page at most

% Paragraph explaining the problam and justifying the system (prevalence of lung cancer, usefullness of screenings and early detection and such)

% Paragraph explaining the purpose of the system.

% Paragraph explaining our general methods (eh aanpak?) and source for inspiration.

\section{Methods}
% Overview, description of the techniques you used, motivate your system design (max 4 pages)

	\subsection{Overview}
	% High level overview of our system. 

	% Basically a flow-chart with a little explenation / motivation.

	\subsection{Preprocessing}

	% Short, resampling + lungmask.

	\subsection{Candidate Selection}

	% Simple thresholding and probably a bit about the other stuff we tried (as part of motivation).

	\subsection{Features}

	% List of features + source. Maybe including a graph or example. Basic phillosophy more = better.

	\subsection{Classification}

	% Subject folded grid-search. 3 different algorithms tried. Motivation ( = keep it relatively simple?).


\section{Results}
% Present FROC curves, show some examples, make comparisons to show how effective subsections of your system are, etc.

	\subsection{Candidate Selection}

	% Performance of candidate selection. Show some examples of stuff missed.

	\subsection{Classification}

	% Performance on training set as measured by ourselves. Show some examples of misclassifications.

	\subsection{ANODE Evaluation}
	% Mandatory: Evaluation on ANODE database and comparison with the existing algorithms. 

	% Performance on test set as measeured by the ANODE people.

\section{Discussion}
% What are weaknesses and strengths of your system.  (1 page max)

% Paragraph per system part.

% Image preprocessing. Lungmask not-ideal (not really bad either), better (or seperate) way of handling areas near the lung membrance could result in a marked improvement. Resampling increases time-performance.

% Candidate Selection. Quick. Not that great. Undermines the geometric features. Point of improvement: some form of vessel extraction.

% Features. Some were good. Some weren't. The pipeline was set up in such a way that adding more would be relatively easy.

% Classification. Features in a csv file means we are really flexible in how we implement classification. Use of scikitlearn means we could easily change things up and try different algorithms.


\bibliography{references}{}
\addcontentsline{toc}{section}{References}
\bibliographystyle{apalike}

\end{document}
